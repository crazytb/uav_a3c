% Introduction Section
% This file can be included in the main paper.tex using % Introduction Section
% This file can be included in the main paper.tex using % Introduction Section
% This file can be included in the main paper.tex using % Introduction Section
% This file can be included in the main paper.tex using \input{sections/introduction}

The proliferation of Unmanned Aerial Vehicles (UAVs) in various applications has created new opportunities for distributed computing and task offloading in edge computing environments. As UAV swarms become increasingly sophisticated, the challenge of optimizing computational task distribution among multiple agents becomes critical for system performance and energy efficiency.

Modern UAV systems are required to process computationally intensive tasks such as real-time image processing, path planning, and sensor data analysis while operating under strict energy and latency constraints. The limited computational resources onboard individual UAVs necessitate intelligent task offloading strategies that can dynamically distribute workloads across the network, including mobile edge computing (MEC) servers and cloud infrastructure.

Reinforcement Learning (RL), particularly deep reinforcement learning approaches, has emerged as a promising solution for addressing the complex decision-making challenges in multi-UAV systems. Among various RL algorithms, Asynchronous Advantage Actor-Critic (A3C) has shown significant potential due to its ability to handle continuous action spaces and its inherent parallelization capabilities. The asynchronous nature of A3C allows multiple agents to learn simultaneously, making it particularly suitable for distributed UAV scenarios where coordination and communication constraints exist.

However, a fundamental question remains regarding the optimal training strategy for A3C in multi-UAV environments: should the system employ a centralized approach with a shared global model, or would decentralized training with individual worker specialization yield superior performance? This question becomes even more critical when considering the heterogeneous nature of operational environments that UAV systems encounter.

This paper investigates two fundamental training strategies for A3C-based multi-UAV task offloading: (1) centralized training using a global model shared across all workers, and (2) decentralized training where individual workers develop specialized policies. Through comprehensive experimental evaluation across multiple environmental configurations, we analyze the performance characteristics, convergence properties, and generalization capabilities of each approach.

The proliferation of Unmanned Aerial Vehicles (UAVs) in various applications has created new opportunities for distributed computing and task offloading in edge computing environments. As UAV swarms become increasingly sophisticated, the challenge of optimizing computational task distribution among multiple agents becomes critical for system performance and energy efficiency.

Modern UAV systems are required to process computationally intensive tasks such as real-time image processing, path planning, and sensor data analysis while operating under strict energy and latency constraints. The limited computational resources onboard individual UAVs necessitate intelligent task offloading strategies that can dynamically distribute workloads across the network, including mobile edge computing (MEC) servers and cloud infrastructure.

Reinforcement Learning (RL), particularly deep reinforcement learning approaches, has emerged as a promising solution for addressing the complex decision-making challenges in multi-UAV systems. Among various RL algorithms, Asynchronous Advantage Actor-Critic (A3C) has shown significant potential due to its ability to handle continuous action spaces and its inherent parallelization capabilities. The asynchronous nature of A3C allows multiple agents to learn simultaneously, making it particularly suitable for distributed UAV scenarios where coordination and communication constraints exist.

However, a fundamental question remains regarding the optimal training strategy for A3C in multi-UAV environments: should the system employ a centralized approach with a shared global model, or would decentralized training with individual worker specialization yield superior performance? This question becomes even more critical when considering the heterogeneous nature of operational environments that UAV systems encounter.

This paper investigates two fundamental training strategies for A3C-based multi-UAV task offloading: (1) centralized training using a global model shared across all workers, and (2) decentralized training where individual workers develop specialized policies. Through comprehensive experimental evaluation across multiple environmental configurations, we analyze the performance characteristics, convergence properties, and generalization capabilities of each approach.

The proliferation of Unmanned Aerial Vehicles (UAVs) in various applications has created new opportunities for distributed computing and task offloading in edge computing environments. As UAV swarms become increasingly sophisticated, the challenge of optimizing computational task distribution among multiple agents becomes critical for system performance and energy efficiency.

Modern UAV systems are required to process computationally intensive tasks such as real-time image processing, path planning, and sensor data analysis while operating under strict energy and latency constraints. The limited computational resources onboard individual UAVs necessitate intelligent task offloading strategies that can dynamically distribute workloads across the network, including mobile edge computing (MEC) servers and cloud infrastructure.

Reinforcement Learning (RL), particularly deep reinforcement learning approaches, has emerged as a promising solution for addressing the complex decision-making challenges in multi-UAV systems. Among various RL algorithms, Asynchronous Advantage Actor-Critic (A3C) has shown significant potential due to its ability to handle continuous action spaces and its inherent parallelization capabilities. The asynchronous nature of A3C allows multiple agents to learn simultaneously, making it particularly suitable for distributed UAV scenarios where coordination and communication constraints exist.

However, a fundamental question remains regarding the optimal training strategy for A3C in multi-UAV environments: should the system employ a centralized approach with a shared global model, or would decentralized training with individual worker specialization yield superior performance? This question becomes even more critical when considering the heterogeneous nature of operational environments that UAV systems encounter.

This paper investigates two fundamental training strategies for A3C-based multi-UAV task offloading: (1) centralized training using a global model shared across all workers, and (2) decentralized training where individual workers develop specialized policies. Through comprehensive experimental evaluation across multiple environmental configurations, we analyze the performance characteristics, convergence properties, and generalization capabilities of each approach.

The proliferation of Unmanned Aerial Vehicles (UAVs) in various applications has created new opportunities for distributed computing and task offloading in edge computing environments. As UAV swarms become increasingly sophisticated, the challenge of optimizing computational task distribution among multiple agents becomes critical for system performance and energy efficiency.

Modern UAV systems are required to process computationally intensive tasks such as real-time image processing, path planning, and sensor data analysis while operating under strict energy and latency constraints. The limited computational resources onboard individual UAVs necessitate intelligent task offloading strategies that can dynamically distribute workloads across the network, including mobile edge computing (MEC) servers and cloud infrastructure.

Reinforcement Learning (RL), particularly deep reinforcement learning approaches, has emerged as a promising solution for addressing the complex decision-making challenges in multi-UAV systems. Among various RL algorithms, Asynchronous Advantage Actor-Critic (A3C) has shown significant potential due to its ability to handle continuous action spaces and its inherent parallelization capabilities. The asynchronous nature of A3C allows multiple agents to learn simultaneously, making it particularly suitable for distributed UAV scenarios where coordination and communication constraints exist.

However, a fundamental question remains regarding the optimal training strategy for A3C in multi-UAV environments: should the system employ a centralized approach with a shared global model, or would decentralized training with individual worker specialization yield superior performance? This question becomes even more critical when considering the heterogeneous nature of operational environments that UAV systems encounter.

This paper investigates two fundamental training strategies for A3C-based multi-UAV task offloading: (1) centralized training using a global model shared across all workers, and (2) decentralized training where individual workers develop specialized policies. Through comprehensive experimental evaluation across multiple environmental configurations, we analyze the performance characteristics, convergence properties, and generalization capabilities of each approach.